% Options for packages loaded elsewhere
\PassOptionsToPackage{unicode}{hyperref}
\PassOptionsToPackage{hyphens}{url}
%
\documentclass[
]{article}
\usepackage{amsmath,amssymb}
\usepackage{lmodern}
\usepackage{iftex}
\ifPDFTeX
  \usepackage[T1]{fontenc}
  \usepackage[utf8]{inputenc}
  \usepackage{textcomp} % provide euro and other symbols
\else % if luatex or xetex
  \usepackage{unicode-math}
  \defaultfontfeatures{Scale=MatchLowercase}
  \defaultfontfeatures[\rmfamily]{Ligatures=TeX,Scale=1}
\fi
% Use upquote if available, for straight quotes in verbatim environments
\IfFileExists{upquote.sty}{\usepackage{upquote}}{}
\IfFileExists{microtype.sty}{% use microtype if available
  \usepackage[]{microtype}
  \UseMicrotypeSet[protrusion]{basicmath} % disable protrusion for tt fonts
}{}
\makeatletter
\@ifundefined{KOMAClassName}{% if non-KOMA class
  \IfFileExists{parskip.sty}{%
    \usepackage{parskip}
  }{% else
    \setlength{\parindent}{0pt}
    \setlength{\parskip}{6pt plus 2pt minus 1pt}}
}{% if KOMA class
  \KOMAoptions{parskip=half}}
\makeatother
\usepackage{xcolor}
\usepackage[margin=1in]{geometry}
\usepackage{color}
\usepackage{fancyvrb}
\newcommand{\VerbBar}{|}
\newcommand{\VERB}{\Verb[commandchars=\\\{\}]}
\DefineVerbatimEnvironment{Highlighting}{Verbatim}{commandchars=\\\{\}}
% Add ',fontsize=\small' for more characters per line
\usepackage{framed}
\definecolor{shadecolor}{RGB}{248,248,248}
\newenvironment{Shaded}{\begin{snugshade}}{\end{snugshade}}
\newcommand{\AlertTok}[1]{\textcolor[rgb]{0.94,0.16,0.16}{#1}}
\newcommand{\AnnotationTok}[1]{\textcolor[rgb]{0.56,0.35,0.01}{\textbf{\textit{#1}}}}
\newcommand{\AttributeTok}[1]{\textcolor[rgb]{0.77,0.63,0.00}{#1}}
\newcommand{\BaseNTok}[1]{\textcolor[rgb]{0.00,0.00,0.81}{#1}}
\newcommand{\BuiltInTok}[1]{#1}
\newcommand{\CharTok}[1]{\textcolor[rgb]{0.31,0.60,0.02}{#1}}
\newcommand{\CommentTok}[1]{\textcolor[rgb]{0.56,0.35,0.01}{\textit{#1}}}
\newcommand{\CommentVarTok}[1]{\textcolor[rgb]{0.56,0.35,0.01}{\textbf{\textit{#1}}}}
\newcommand{\ConstantTok}[1]{\textcolor[rgb]{0.00,0.00,0.00}{#1}}
\newcommand{\ControlFlowTok}[1]{\textcolor[rgb]{0.13,0.29,0.53}{\textbf{#1}}}
\newcommand{\DataTypeTok}[1]{\textcolor[rgb]{0.13,0.29,0.53}{#1}}
\newcommand{\DecValTok}[1]{\textcolor[rgb]{0.00,0.00,0.81}{#1}}
\newcommand{\DocumentationTok}[1]{\textcolor[rgb]{0.56,0.35,0.01}{\textbf{\textit{#1}}}}
\newcommand{\ErrorTok}[1]{\textcolor[rgb]{0.64,0.00,0.00}{\textbf{#1}}}
\newcommand{\ExtensionTok}[1]{#1}
\newcommand{\FloatTok}[1]{\textcolor[rgb]{0.00,0.00,0.81}{#1}}
\newcommand{\FunctionTok}[1]{\textcolor[rgb]{0.00,0.00,0.00}{#1}}
\newcommand{\ImportTok}[1]{#1}
\newcommand{\InformationTok}[1]{\textcolor[rgb]{0.56,0.35,0.01}{\textbf{\textit{#1}}}}
\newcommand{\KeywordTok}[1]{\textcolor[rgb]{0.13,0.29,0.53}{\textbf{#1}}}
\newcommand{\NormalTok}[1]{#1}
\newcommand{\OperatorTok}[1]{\textcolor[rgb]{0.81,0.36,0.00}{\textbf{#1}}}
\newcommand{\OtherTok}[1]{\textcolor[rgb]{0.56,0.35,0.01}{#1}}
\newcommand{\PreprocessorTok}[1]{\textcolor[rgb]{0.56,0.35,0.01}{\textit{#1}}}
\newcommand{\RegionMarkerTok}[1]{#1}
\newcommand{\SpecialCharTok}[1]{\textcolor[rgb]{0.00,0.00,0.00}{#1}}
\newcommand{\SpecialStringTok}[1]{\textcolor[rgb]{0.31,0.60,0.02}{#1}}
\newcommand{\StringTok}[1]{\textcolor[rgb]{0.31,0.60,0.02}{#1}}
\newcommand{\VariableTok}[1]{\textcolor[rgb]{0.00,0.00,0.00}{#1}}
\newcommand{\VerbatimStringTok}[1]{\textcolor[rgb]{0.31,0.60,0.02}{#1}}
\newcommand{\WarningTok}[1]{\textcolor[rgb]{0.56,0.35,0.01}{\textbf{\textit{#1}}}}
\usepackage{graphicx}
\makeatletter
\def\maxwidth{\ifdim\Gin@nat@width>\linewidth\linewidth\else\Gin@nat@width\fi}
\def\maxheight{\ifdim\Gin@nat@height>\textheight\textheight\else\Gin@nat@height\fi}
\makeatother
% Scale images if necessary, so that they will not overflow the page
% margins by default, and it is still possible to overwrite the defaults
% using explicit options in \includegraphics[width, height, ...]{}
\setkeys{Gin}{width=\maxwidth,height=\maxheight,keepaspectratio}
% Set default figure placement to htbp
\makeatletter
\def\fps@figure{htbp}
\makeatother
\setlength{\emergencystretch}{3em} % prevent overfull lines
\providecommand{\tightlist}{%
  \setlength{\itemsep}{0pt}\setlength{\parskip}{0pt}}
\setcounter{secnumdepth}{5}
\ifLuaTeX
  \usepackage{selnolig}  % disable illegal ligatures
\fi
\IfFileExists{bookmark.sty}{\usepackage{bookmark}}{\usepackage{hyperref}}
\IfFileExists{xurl.sty}{\usepackage{xurl}}{} % add URL line breaks if available
\urlstyle{same} % disable monospaced font for URLs
\hypersetup{
  pdftitle={Evidencia 1: Análisis inicial},
  hidelinks,
  pdfcreator={LaTeX via pandoc}}

\title{Evidencia 1: Análisis inicial}
\author{}
\date{\vspace{-2.5em}}

\begin{document}
\maketitle

{
\setcounter{tocdepth}{2}
\tableofcontents
}
\hypertarget{instituto-tecnoluxf3gico-de-estudios-superiores-de-monterrey}{%
\section{Instituto Tecnológico de Estudios Superiores de
Monterrey}\label{instituto-tecnoluxf3gico-de-estudios-superiores-de-monterrey}}

\hypertarget{anuxe1lisis-de-biologuxeda-computacional---bt1013}{%
\subsection{Análisis de biología computacional -
BT1013}\label{anuxe1lisis-de-biologuxeda-computacional---bt1013}}

\hypertarget{desiruxe9e-espinosa-contreras---a01425162}{%
\subsubsection{Desirée Espinosa Contreras -
A01425162}\label{desiruxe9e-espinosa-contreras---a01425162}}

\hypertarget{dulce-nahomi-bucio-rivas---a01425284}{%
\subsubsection{Dulce Nahomi Bucio Rivas -
A01425284}\label{dulce-nahomi-bucio-rivas---a01425284}}

\hypertarget{introducciuxf3n}{%
\subsection{Introducción}\label{introducciuxf3n}}

La pandemia de COVID-19, provocada por el virus SARS-CoV-2, tuvo su
inicio en el año 2020, cuando la Organización Mundial de la Salud
declaró la epidemia originalmente situada en China como una emergencia
de salud internacional, para posteriormente ser declarada como pandemia.
En la actualidad, la pandemia continúa en un estado activo, habiendo
alcanzado una cifra estimada de 676,609,955 casos de coronavirus a nivel
mundial, con un total de 6,881,955 lamentables muertes, según la
Universidad Johns Hopkins. Asimismo, en México, se estima un total de
7,483,444 casos diagnosticados y 333,188 muertes por COVID-19.

Para nuestro enfoque de estudio es importante definir qué es una
variante, la cual es un genoma viral (código genético) que puede incluir
una o más mutaciones. La primera variación detectada fue la Alfa
B.1.1.7, la cual fue observada por primera vez en el suroeste de
Inglaterra y se caracteriza por ser entre 30\% y 70\% más contagiosa que
el virus original.

Existen variables características descubiertas en distintas regiones del
mundo que fueron principalmente bajo observación, tales son:\\
\textbf{Variante Alfa B.1.1.7}: Descubierta en Reino Unido, fue la
variable dominante a nivel mundial hasta el año 2021 con una frecuencia
mayor al 80\% en regiones tales como Europa y América del Norte.

\textbf{Variante Beta B.1.351}: Detectada en Sudáfrica, siendo
mayoritariamente dominante en esta región pero en otras no rebasando una
frecuencia del 20\%. Caracterizada únicamente por tener mayor carga
vírica pero no se encontró ni se demostró nada más acerca de su
transmisibilidad.

\textbf{Variante Gamma P1}: Procedente de la Amazonas Brasileña se
convirtió en la más dominante en América del Sur mientras que en otras
regiones la frecuencia no supera el 10\%, siendo caracterizada por ser
más contagiosa.

\textbf{Variante DELTA B.1.617.2}: Encontrada en la India, esta variante
dio señales de ser más contagiosa y resistente a vacunas y tratamientos.
Fue la variante responsable del colapso del sistema sanitario en ese
país, así como de su propagación por 26 países de la región europea de
la OMS.

\hypertarget{propuesta}{%
\subsection{Propuesta}\label{propuesta}}

\hypertarget{para-el-proyecto-tenemos-planeado-hacer-un-anuxe1lisis-de-las-secuencias-del-virus-en-muxe9xico-de-febrero-a-diciembre-de-2020-y-de-febrero-a-diciembre-de-2021.-esto-con-el-objetivo-de-encontrar-los-cambios-que-experimentuxf3-el-sars-cov-2-en-su-estructura-durante-estos-periodos-de-tiempo.-analizaremos-principalmente-los-genes-s-m-e-y-n-los-cuales-han-demostrado-tener-un-papel-importante-en-la-construcciuxf3n-de-la-estructura-del-virus-especialmente-el-gen-s.-a-partir-de-los-resultados-que-obtengamos-se-podruxe1n-extraer-diversas-conclusiones-por-ejemplo-si-encontramos-una-cantidad-relevante-de-mutaciones-en-el-gen-s-entonces-podruxeda-implicarse-que-dichas-mutaciones-estuvieron-involucradas-en-el-aumento-de-la-transmisibilidad-del-virus-yo-en-la-efectividad-de-las-vacunas.-ademuxe1s-podruxedamos-obtener-informaciuxf3n-valiosa-sobre-la-evoluciuxf3n-del-virus.}{%
\subsection{Para el proyecto, tenemos planeado hacer un análisis de las
secuencias del virus en México de febrero a diciembre de 2020, y de
febrero a diciembre de 2021. Esto con el objetivo de encontrar los
cambios que experimentó el SARS-CoV-2 en su estructura durante estos
periodos de tiempo. Analizaremos principalmente los genes S, M, E y N,
los cuales han demostrado tener un papel importante en la construcción
de la estructura del virus, especialmente el gen S. A partir de los
resultados que obtengamos, se podrán extraer diversas conclusiones, por
ejemplo, si encontramos una cantidad relevante de mutaciones en el gen
S, entonces podría implicarse que dichas mutaciones estuvieron
involucradas en el aumento de la transmisibilidad del virus y/o en la
efectividad de las vacunas. Además, podríamos obtener información
valiosa sobre la evolución del
virus.}\label{para-el-proyecto-tenemos-planeado-hacer-un-anuxe1lisis-de-las-secuencias-del-virus-en-muxe9xico-de-febrero-a-diciembre-de-2020-y-de-febrero-a-diciembre-de-2021.-esto-con-el-objetivo-de-encontrar-los-cambios-que-experimentuxf3-el-sars-cov-2-en-su-estructura-durante-estos-periodos-de-tiempo.-analizaremos-principalmente-los-genes-s-m-e-y-n-los-cuales-han-demostrado-tener-un-papel-importante-en-la-construcciuxf3n-de-la-estructura-del-virus-especialmente-el-gen-s.-a-partir-de-los-resultados-que-obtengamos-se-podruxe1n-extraer-diversas-conclusiones-por-ejemplo-si-encontramos-una-cantidad-relevante-de-mutaciones-en-el-gen-s-entonces-podruxeda-implicarse-que-dichas-mutaciones-estuvieron-involucradas-en-el-aumento-de-la-transmisibilidad-del-virus-yo-en-la-efectividad-de-las-vacunas.-ademuxe1s-podruxedamos-obtener-informaciuxf3n-valiosa-sobre-la-evoluciuxf3n-del-virus.}}

\hypertarget{desarrollo}{%
\subsection{Desarrollo}\label{desarrollo}}

Iniciaremos el análisis de las secuencias, utilizando las diferentes
herramientas que nos ofrece R.

\begin{Shaded}
\begin{Highlighting}[]
\CommentTok{\#Inicializamos la librería que nos permite leer secuencias en formato fasta, también leemos ambas secuencias, tanto de 2020 como de 2021.}
\FunctionTok{library}\NormalTok{(seqinr)}
\FunctionTok{library}\NormalTok{(knitr)}

\NormalTok{mx\_2020 }\OtherTok{=} \FunctionTok{read.fasta}\NormalTok{(}\StringTok{"2020\_sequences\_mx.fasta"}\NormalTok{)}
\NormalTok{mx\_2021 }\OtherTok{=} \FunctionTok{read.fasta}\NormalTok{(}\StringTok{"2021\_sequences\_mx.fasta"}\NormalTok{)}

\NormalTok{mutations }\OtherTok{=} \FunctionTok{data.frame}\NormalTok{(}
  \AttributeTok{mutation =} \FunctionTok{character}\NormalTok{(),}
  \AttributeTok{pos\_global =} \FunctionTok{integer}\NormalTok{(),}
  \AttributeTok{cambioCodon =} \FunctionTok{character}\NormalTok{(),}
  \AttributeTok{cambioAmino =} \FunctionTok{character}\NormalTok{(),}
  \AttributeTok{gen =} \FunctionTok{character}\NormalTok{()}
\NormalTok{)}

\NormalTok{aminoacido }\OtherTok{=} \ControlFlowTok{function}\NormalTok{(codon) \{}
\NormalTok{  aminoacid }\OtherTok{=} \ControlFlowTok{switch}\NormalTok{(}
\NormalTok{    codon, }
    \StringTok{"GCU"} \OtherTok{=} \StringTok{"A"}\NormalTok{, }\StringTok{"GCC"} \OtherTok{=} \StringTok{"A"}\NormalTok{, }\StringTok{"GCA"} \OtherTok{=} \StringTok{"A"}\NormalTok{, }\StringTok{"GCG"} \OtherTok{=} \StringTok{"A"}\NormalTok{,}
    \StringTok{"CGU"} \OtherTok{=} \StringTok{"R"}\NormalTok{, }\StringTok{"CGC"} \OtherTok{=} \StringTok{"R"}\NormalTok{, }\StringTok{"CGA"} \OtherTok{=} \StringTok{"R"}\NormalTok{, }\StringTok{"CGG"} \OtherTok{=} \StringTok{"R"}\NormalTok{,}
    \StringTok{"AGA"} \OtherTok{=} \StringTok{"R"}\NormalTok{, }\StringTok{"AGG"} \OtherTok{=} \StringTok{"R"}\NormalTok{, }\StringTok{"AAU"} \OtherTok{=} \StringTok{"N"}\NormalTok{, }\StringTok{"AAC"} \OtherTok{=} \StringTok{"N"}\NormalTok{,}
    \StringTok{"GAU"} \OtherTok{=} \StringTok{"D"}\NormalTok{, }\StringTok{"GAC"} \OtherTok{=} \StringTok{"D"}\NormalTok{, }\StringTok{"UGU"} \OtherTok{=} \StringTok{"C"}\NormalTok{, }\StringTok{"UGC"} \OtherTok{=} \StringTok{"C"}\NormalTok{,}
    \StringTok{"CAA"} \OtherTok{=} \StringTok{"Q"}\NormalTok{, }\StringTok{"CAG"} \OtherTok{=} \StringTok{"Q"}\NormalTok{, }\StringTok{"GAA"} \OtherTok{=} \StringTok{"E"}\NormalTok{, }\StringTok{"GAG"} \OtherTok{=} \StringTok{"E"}\NormalTok{,}
    \StringTok{"GGU"} \OtherTok{=} \StringTok{"G"}\NormalTok{, }\StringTok{"GGC"} \OtherTok{=} \StringTok{"G"}\NormalTok{, }\StringTok{"GGA"} \OtherTok{=} \StringTok{"G"}\NormalTok{, }\StringTok{"GGG"} \OtherTok{=} \StringTok{"G"}\NormalTok{,}
    \StringTok{"CAU"} \OtherTok{=} \StringTok{"H"}\NormalTok{, }\StringTok{"CAC"} \OtherTok{=} \StringTok{"H"}\NormalTok{, }\StringTok{"AUU"} \OtherTok{=} \StringTok{"I"}\NormalTok{, }\StringTok{"AUC"} \OtherTok{=} \StringTok{"I"}\NormalTok{,}
    \StringTok{"AUA"} \OtherTok{=} \StringTok{"I"}\NormalTok{, }\StringTok{"UUA"} \OtherTok{=} \StringTok{"L"}\NormalTok{, }\StringTok{"UUG"} \OtherTok{=} \StringTok{"L"}\NormalTok{, }\StringTok{"CUU"} \OtherTok{=} \StringTok{"L"}\NormalTok{,}
    \StringTok{"CUC"} \OtherTok{=} \StringTok{"L"}\NormalTok{, }\StringTok{"CUA"} \OtherTok{=} \StringTok{"L"}\NormalTok{, }\StringTok{"CUG"} \OtherTok{=} \StringTok{"L"}\NormalTok{, }\StringTok{"AAA"} \OtherTok{=} \StringTok{"K"}\NormalTok{,}
    \StringTok{"AAG"} \OtherTok{=} \StringTok{"K"}\NormalTok{, }\StringTok{"AUG"} \OtherTok{=} \StringTok{"M"}\NormalTok{, }\StringTok{"UUU"} \OtherTok{=} \StringTok{"F"}\NormalTok{, }\StringTok{"UUC"} \OtherTok{=} \StringTok{"F"}\NormalTok{,}
    \StringTok{"CCU"} \OtherTok{=} \StringTok{"P"}\NormalTok{, }\StringTok{"CCC"} \OtherTok{=} \StringTok{"P"}\NormalTok{, }\StringTok{"CCA"} \OtherTok{=} \StringTok{"P"}\NormalTok{, }\StringTok{"CCG"} \OtherTok{=} \StringTok{"P"}\NormalTok{,}
    \StringTok{"UCU"} \OtherTok{=} \StringTok{"S"}\NormalTok{, }\StringTok{"UCC"} \OtherTok{=} \StringTok{"S"}\NormalTok{, }\StringTok{"UCA"} \OtherTok{=} \StringTok{"S"}\NormalTok{, }\StringTok{"UCG"} \OtherTok{=} \StringTok{"S"}\NormalTok{,}
    \StringTok{"AGU"} \OtherTok{=} \StringTok{"S"}\NormalTok{, }\StringTok{"AGC"} \OtherTok{=} \StringTok{"S"}\NormalTok{, }\StringTok{"ACU"} \OtherTok{=} \StringTok{"T"}\NormalTok{, }\StringTok{"ACC"} \OtherTok{=} \StringTok{"T"}\NormalTok{,}
    \StringTok{"ACA"} \OtherTok{=} \StringTok{"T"}\NormalTok{, }\StringTok{"ACG"} \OtherTok{=} \StringTok{"T"}\NormalTok{, }\StringTok{"UGG"} \OtherTok{=} \StringTok{"W"}\NormalTok{, }\StringTok{"UAU"} \OtherTok{=} \StringTok{"Y"}\NormalTok{,}
    \StringTok{"UAC"} \OtherTok{=} \StringTok{"Y"}\NormalTok{, }\StringTok{"GUU"} \OtherTok{=} \StringTok{"V"}\NormalTok{, }\StringTok{"GUC"} \OtherTok{=} \StringTok{"V"}\NormalTok{, }\StringTok{"GUA"} \OtherTok{=} \StringTok{"V"}\NormalTok{,}
    \StringTok{"GUG"} \OtherTok{=} \StringTok{"V"}
\NormalTok{  )}
  \FunctionTok{return}\NormalTok{ (aminoacid)}
\NormalTok{\}}
\end{Highlighting}
\end{Shaded}

\begin{center}\rule{0.5\linewidth}{0.5pt}\end{center}

\hypertarget{bibliografuxeda}{%
\paragraph{Bibliografía}\label{bibliografuxeda}}

-- Ferrer, R. (2020). Pandemia por COVID-19: el mayor reto de la
historia del intensivismo. Medicina Intensiva, 44(6), 323-324.
\url{https://doi.org/10.1016/j.medin.2020.04.002}\\
-- COVID-19 Map - Johns Hopkins Coronavirus Resource Center. (s.f.).
Johns Hopkins Coronavirus Resource Center.
\url{https://coronavirus.jhu.edu/map.html}\\
-- Updated working definitions and primary actions for SARSCOV2
variants. (s.f.). Retrieved April 25, 2023, from
\url{https://www.who.int/publications/m/item/historical-working-definitions-and-primary-actions-for-sars-cov-2-variants}\\
-- Pulido, S. (2021). Variantes Covid por países: Un análisis completo y
experiencias en Reino Unido y ee.uu con Los Test Genéticos. Retrieved
April 25, 2023, from
\url{https://economiadelasalud.com/topics/difusion/variantes-covid-por-paises-un-analisis-completo-y-experiencias-en-reino-unido-y-ee-uu-con-los-test-geneticos/}

\end{document}
